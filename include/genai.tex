It is perfectly understandable if you decide to take advantage of modern tools, particularly generative AI, to increase your productivity. However, it is important to consider the following guiding principles:
\begin{compactitem}
	\item \textit{you cannot claim work as yours that you haven't created} (it's plagiarism),
    \item \textit{you are fully responsible} for any mistakes, particularly false or faulty references, incorrect citations, unvalidated statements, etc.,
    \item \textit{your learning experience is more important than fast delivery}. If you really believe an LLM does things better than you could, whether it is coding or writing or brainstorming or anything else, consider
        \begin{inparaenum}[(i)]
            \item whether it is worth acquiring this skill, and
            \item why you are enrolled in an institution of higher education if all you do is prompting.
        \end{inparaenum}
\end{compactitem}

You may use AI tools to:
\begin{compactitem}
	\item proofread your original written work to, e.g., improve language, phrasing, grammar, spelling, or
	\item generate code snippets, documentation, test cases, profile, etc.,
\end{compactitem}
as long as you scrutinize and validate the output.

You may \textbf{not} use AI to:
\begin{compactitem}
	\item write your final paper or significant section of it,
	\item create 'personal' insights or observations, or
	\item generate longer source code sequences.
\end{compactitem}

Unless you state that you used an AI tool, your work will be evaluated under the assumption that no such tools were used, and if it turns out that AI tools were nevertheless used in any non-trivial form, it will be considered plagiarism. 

When you used AI tools, clearly indicate the following:
\begin{compactitem}
	\item \textit{which} tools were used,
	\item \textit{for what} they were used, and 
    \item \textit{how} they were used (prompts) and how the generated output informed or shaped your final submission.
\end{compactitem}
