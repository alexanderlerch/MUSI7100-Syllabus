\documentclass[letterpaper,oneside,10pt]{scrartcl}

\usepackage[margin=1in]{geometry}
\usepackage[T1]{fontenc}
\usepackage[ansinew]{inputenc}
\usepackage{lmodern} %Type1-font for non-english texts and characters
\usepackage{microtype} %Type1-font for non-english texts and characters
\usepackage{graphicx} %%For loading graphic files
\usepackage{amsmath}
\usepackage{amsthm}
\usepackage{amsfonts}
\usepackage{url}
\usepackage{xcolor}
\usepackage{hyperref}

\renewcommand{\familydefault}{\sfdefault}
\setlength{\parindent}{0mm}
 



%%%%%%%%%%%%%%%%%%%%%%%%%%%%%%%%%%%%%%%%%%%%%%%%%%%%%%%%%%%%%
%% DOCUMENT
%%%%%%%%%%%%%%%%%%%%%%%%%%%%%%%%%%%%%%%%%%%%%%%%%%%%%%%%%%%%%
\begin{document}

\title{MUSI 4699/7100: Syllabus}
\author{Music Technology Research Lab --- Music Informatics Group}
\date{Fall 2024} %%If commented, the current date is used.
\maketitle


\pagestyle{plain} %Now display headings: headings / fancy / ...

\section*{Course Details}
    \begin{tabular}{ll}
        \textbf{class time} & seminar M 14:00, group meeting tba, individual meetings tbd \\
        \textbf{location} & seminar WV175, SoA266A\\
        \textbf{credits} & 3 credit hours\\
        \textbf{modality} & in person
    \end{tabular}
    
\section*{Instructor Information}
    \begin{tabular}{lll}
        %\textbf{} & \textbf{instructor} & \textbf{teaching assistant}\\
        \textbf{name} & Alexander Lerch & \\
        \textbf{email} & \url{alexander.lerch@gatech.edu} & \\
        \textbf{location} & SoA 266A (brigde between architecture buildings) & \\
        \textbf{office hours} & {by online appointment (\href{https://outlook.office.com/bookwithme/user/fd6fbed37b7c4ee8a3e25e7cdcbee3f5@gatech.edu?anonymous&ep=plink}{bookwithme online appointment})} & \\
    \end{tabular}

\section*{Group Communication}
    You will receive invitation links to the mailing lists below --- notify me if you don't or if you have any questions or concerns.
    
    \vspace{\baselineskip}
    \begin{tabular}{ll}
        \textbf{MS Teams Channels} & \\
        \textbf{musicinformatics@googlegroups.com} & internal: meetups, discussions, etc.  \\
        \textbf{music-informatics-alumni@googlegroups.com} & external: announcements \& accomplishments for all alumni  \\
        %\textbf{music-performance-analysis@googlegroups.com} & internal: FBA \& funded research 
    \end{tabular}
        
\section{General Information}        
    \subsection{Course Description}
        Guided research and creative work in music technology. Investigation of novel technological and artistic concepts. Design and develop new hardware, software, and musical artifacts.
        
    \subsection{Prerequisites}
        There are no formal prerequisites. Prior coursework in signals and systems and machine learning is expected. Programming experience and familiarity with Python will be helpful.

    \subsection{Learning Outcomes}
        After successful completion of the class, the students will demonstrate 
        \begin{itemize}
            \item   knowledge of state of the art algorithms and technology in music technology,
            \item   the ability to design and implement algorithms, and to design and implement systems for the analysis, synthesis, and/or processing of audio and music,
            \item   the ability to design and execute a systematic and meaningful evaluation methodology including appropriate selection and handling of data, evaluation procedure, and identification of suitable metrics,
            \item   the ability to plan for all stages of a research project (i.e., formulating the research question, doing a literature survey, finding/implementing a baseline/comparison, designing the experiment, evaluating and analyzing the results systematically), and 
            \item   the ability to present the results in a structured and clear, understandable way in scientific writing and a poster presentation.
        \end{itemize}
        
    \subsection{Course Modality Information}
        Personal \textbf{attendance is mandatory} in  the Monday seminar series (2pm) and the group meeting (TBA) unless announced otherwise. The individual meetings are by default scheduled in person but can be adapted to other formats if agreed on.
        
\section{Course Requirements \& Grading}
    The overall grade consists of:
    \smallskip
    
    \begin{tabular}{lll}
        \textbf{attendance} & 15\%\\
        \textbf{project work} & 50\%\\
        \textbf{paper} & 15\%\\
        \textbf{final (poster) presentation} & 20\% \\
    \end{tabular}
    Undergraduate students will be graded identically but will receive a grade 10\% higher.

    \subsection{Description of Graded Components}
        \begin{itemize}
            \item   \textbf{attendance}:\\ participation in Monday seminars and in group meetings as well as in scheduled individual meetings (all mandatory)
            \item   \textbf{project work}:\\ your individual or group project for this semester. Grades will assigned according to quality, significance, and impact of your contributions to the project, the ability to complete assigned tasks in a timely manner, the ability to communicate and collaborate with others, and the innovations and ideas that shape the direction of our research.
            \item   \textbf{paper} (due date: before the final instructional class days):\\ conference style paper describing the project in a scientific style in a quality that would allow conference submission. Grades will be assigned according to structure, clarity, references, quality of information, and form. In addition, second year students will also provide a MS project proposal in a similar format at the begin of their 3rd semester.
            \item   \textbf{presentation}:\\ there will be a research showcase at the end of the semester (assigned slot during final exam week) where you will present your work to other students and visitors. Second year students will also present their MS project or thesis in the Monday seminars. The presentation grade will be assigned according to the organization, completeness, verbal and non-verbal presentation skills, and the quality of the visual materials (structure, easy to read, visualizations, etc.).
        \end{itemize}
        
    \subsection{Grading and Grading Policies}
        All graded components will be graded in points. The final grade for the course will be determined by dividing the total points earned by the number of points possible for each of the categories listed above. 

These numbers will be converted into a letter grade according to the following scale: 
    \smallskip
    
\begin{tabular}{ll}
    \textbf{A} & 100--90\%\\
    \textbf{B} & 89--80\%\\
    \textbf{C} & 79--70\%\\
    \textbf{D} & 69--60\%\\
    \textbf{F} & 59\% and below 
\end{tabular}
    \smallskip
    

Grades may be assigned per group or individually as announced (e.g., projects are in some cases per group, quizzes are usually per individual).
 

    \section{Course Materials}

        \subsection{Software}
                Assignments are due in the language announced (commonly python). The project work can be done in any programming language approved after discussion with the instructor. The most common choices are Python and Matlab. Matlab is accessible at \url{www.matlab.gatech.edu}.
    
    With respect to tools, \textbf{prepare to use github} (github.gatech.edu or github.com or some other version control system). I recommend using the github issues in connection with milestones to keep track of your project progress. Other recommended tools are
    \begin{itemize}
        \item   Zotero for bibliography management, and
        \item   \LaTeX\ for scientific typesetting.
    \end{itemize}
 
            
        \subsection{Course Management}
            The class will be managed through Canvas.
            
    %\section{Method of Evaluation}
        %The overall grade consists of:
        %\begin{itemize}
            %\item   \textbf{10\% }
            %\item   \textbf{50\% }
            %\item   \textbf{20\% }
            %\item   \textbf{20\% }
        %\end{itemize}
        %
        %%The winners of the video competition will receive additional points to be \textit{added to their final grade} (1st: 8 points, 2nd: 5 points, 3rd: 3 points).
        %
        %\subsection{Grading and Grading Policies}
            %All graded components will be graded in points. The final grade for the course will be determined by dividing the total points earned by the number of points possible for each of the categories listed above. 

These numbers will be converted into a letter grade according to the following scale: 
    \smallskip
    
\begin{tabular}{ll}
    \textbf{A} & 100--90\%\\
    \textbf{B} & 89--80\%\\
    \textbf{C} & 79--70\%\\
    \textbf{D} & 69--60\%\\
    \textbf{F} & 59\% and below 
\end{tabular}
    \smallskip
    

Grades may be assigned per group or individually as announced (e.g., projects are in some cases per group, quizzes are usually per individual).
 
\section{Course Schedule}
    The class schedule is based on weekly meetings: the weekly Monday seminar, weekly group meetings, as well as weekly individual meetings. Regular individual meetings (20 minutes) will be scheduled at the begin of the semester, time-slots will be available for additional (sign-up) appointments.
    
    Relevant information is available in our github repo: \url{https://github.gatech.edu/alerch3/MUSI7100-Projects}
    %
    Other information can be found in this repository as well, such as
    \begin{itemize}
        \item   lab meeting schedule:\\ \url{https://github.gatech.edu/alerch3/MUSI7100-Projects/blob/master/lab-meeting.md}
        \item   general info:\\ \url{https://github.gatech.edu/alerch3/MUSI7100-Projects/blob/master/readme.md}
        \item   GPU overview:\\ \url{https://github.gatech.edu/alerch3/MUSI7100-Projects/blob/master/lab-machines.md}
        \item   templates for slides and posters
    \end{itemize}
    
    %    Since all classes do not progress at the same rate, it may be necessary to modify the above schedule as circumstances dictate. For example, the number and frequency of assignments may be altered or the schedule of the classes may be changed. In either of these cases, adequate notification will be given and be discussed in class.
    


    
\section{Course Expectations \& Guidelines}
    
    \subsection{Academic Integrity}
        Georgia Tech aims to cultivate a community based on trust, academic integrity, and honor. Students are expected to act according to the highest ethical standards.  For information on Georgia Tech's Academic Honor Code, please visit  or .

        \begin{itemize}
            \item \url{http://www.catalog.gatech.edu/policies/honor-code/} or
            \item \url{http://www.catalog.gatech.edu/rules/18/}.
        \end{itemize}
        
Any student suspected of cheating or plagiarizing on a quiz, exam, or assignment will be reported to the Office of Student Integrity, who will investigate the incident and identify the appropriate penalty for violations.

The use of generative machine learning systems for code and text generation is strongly discouraged as it negates some of the learning outcomes of this class. Note that the student takes full responsibility for factual errors and potential plagiarism in generated text and code.


    \subsection{Accommodations for Individuals with Disabilities}
                If you are a student with learning needs that require special accommodation, contact the Office of Disability Services (often referred to as ADAPTS) at (404)894-2563 or 
        \begin{itemize}
            \item \url{http://disabilityservices.gatech.edu}
        \end{itemize} 
        as soon as possible to make an appointment to discuss your special needs and to obtain an accommodations letter. Please also e-mail me as soon as possible in order to set up a time to discuss your learning needs.

        
    %\subsection{Active Participation}
        %        You are expected to attend the sessions unless you have a compelling reason not to do so. In any case active participation either synchronously or asynchronously is expected.
    
    %


    %\subsection{Guthman Competition}
        %You are required to attend the Guthman competition concert on Saturday, March 9th at 7 pm at the Ferst Center ---for which you will receive a complimentary ticket--- and to sign up for a shift to assist at the competition on March 7, 8, or 9. We also strongly encourage you to attend other Guthman events on March 8 and 9, including presentations by our finalists, a panel with our judges, and a music/art/tech fair at which you can share your own work. More details will be available later in the semester.
        
    \subsection{Digital Etiquette}
                As a significant part of our communication may remain virtual, I appreciate your cooperation. Please make extensive use of chat and mailing lists to stay in touch with each other. \textbf{Check both Teams and email at least once every business day}.
        Try to actively reach out to me and to your colleagues to help or ask for help. In any remote meetings,
        \begin{itemize}
            \item   please leave your camera on and, in case of more than 4 participants, your microphone off by default,
            \item   feel free to either just speak up or use the chat or raised hands to interrupt.
        \end{itemize}

        
    \subsection{Extensions, Late Assignments, Missed Exams}
                All assignments, papers, and other artifacts are due \textbf{ON THE DUE DATE}. The due date will be announced per assignment/task on t-square. A penalty of \textbf{TEN POINTS PER 24~HOURS} will be applied to all late assignments/tasks and late project papers. Documented illnesses and family emergencies are excepted. Quizzes and exams cannot be made up unless you have a valid, documented excuse.


    \subsection{Student Use of Mobile Devices in the Classroom}
                The use of mobile devices in the classroom is prohibited unless explicitly allowed by the instructor.

        
    \subsection{Student-Faculty Expectations}
                At Georgia Tech we believe that it is important to continually strive for an atmosphere of mutual respect, acknowledgment, and responsibility between faculty members and the student body. See 
        \begin{itemize}
            \item \url{http://www.catalog.gatech.edu/rules/22}
        \end{itemize} 
        for an articulation of some basic expectations --- that you can have of me, and that I have of you. In the end, simple respect for knowledge, hard work, and cordial interactions will help build the environment we seek. Therefore, I encourage you to remain committed to the ideals of Georgia Tech while in this class.

        
    \subsection{Diversity}
        The School of Music community of faculty, staff, and students aspires to create and nurture an environment that is supportive of all backgrounds where different views and ideas are respected and encouraged. In all our pursuits, we commit to justice, diversity, equity, and inclusion with regard to race, national origin, language, age, sexual orientation, gender, religion, and ability. Moreover, we will encourage intellectual inquiry and respectful exchange that cements our dedication to these principles.
        
    \subsection{Academic Grievance Policy}
        Students should first discuss any concerns with the relevant faculty member; if it is not possible to come to resolution with the faculty member, students may then report the matter to the appropriate administrator (Chair or Associate Chair or Director of Studies) of the department of instruction or report it here: 
        \begin{itemize}
            \item \url{http://www.contact.gatech.edu/academicgrievance}
        \end{itemize} 
    The GT grievance policy can be found at 
        \begin{itemize}
            \item \url{https://provost.gatech.edu/reporting-units/conflict-resolution-ombuds/academic-grievance-policy}. 
        \end{itemize} 
    
    %Additionally, if you need formal assistance, please contact Associate Provost Jennifer Herazy (\url{mailto:herazy@gatech.edu}). For informal assistance or to speak to someone who can be a sounding board for you, please contact one of the Ombuds staff: Russ Callen (\url{mailto:russ.callen@ece.gatech.edu}) or Leigh Bottomley (\url{leigh.bottomley@gatech.edu}).
\end{document}

